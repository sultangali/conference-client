\documentclass[12pt]{article}

\usepackage{libertine}
\usepackage[T1]{fontenc}
\usepackage[utf8]{inputenc}
\usepackage{microtype}
\usepackage[english, russian]{babel}
\usepackage{amsmath}
\usepackage{graphicx}
\usepackage{geometry}
\usepackage{fancyhdr}
\usepackage{natbib}
\usepackage{xcolor}
\usepackage{titlesec}
\usepackage{booktabs}
\usepackage[colorlinks=true,linkcolor=black,citecolor=black,urlcolor=black]{hyperref}

\geometry{
    margin=1in,
    headheight=12pt
}
\setcitestyle{square}

\pagestyle{fancy}
\fancyhf{}
\renewcommand{\headrulewidth}{0.4pt}
\renewcommand{\footrulewidth}{0.4pt}
\fancyhead[L]{\small Актуальные вопросы междисциплинарных научных исследований}
\fancyhead[R]{\small\thepage}
\fancyfoot[C]{\small Караганда, 2025}

\titleformat{\section}
    {\normalfont\large\bfseries}{\thesection}{1em}{}
\titlespacing*{\section}
    {0pt}{2.5ex plus 1ex minus .2ex}{1.5ex plus .2ex}

\makeatletter
\renewcommand{\maketitle}{%
    \begin{center}
        \Large\@title
        
        \vspace{0.4cm}
        \large\@author
        
        \vspace{0.5cm}
        \normalsize\textit{\textsuperscript{1}Название университета, город, страна\\
          E-mail: test@example.com\\
          \textsuperscript{2}Название университета, город, страна\\
          E-mail: test@example.com\\
          \textsuperscript{3}Название университета, город, страна\\
          E-mail: test@example.com\\}
    \end{center}
}
\makeatother

% Paper info
\title{НАЗВАНИЕ ВАШЕГО ТЕЗИСА}

% Uncomment and update the following line upon acceptance
% \author{Author Name(s)\\ % Institution Name(s)}

%Comment the following line upon acceptance
\author{Автор-корреспондент\textsuperscript{1}, Второй автор\textsuperscript{2}, Третий автор\textsuperscript{3}}


\date{}

\begin{document}

\maketitle

С этого момента начитается текст вашего тезиса...

Шаблон используется для форматирования вашего тезиса и оформления текста. Все поля, ширина колонок, межстрочные интервалы и шрифты текста предписаны; пожалуйста, не изменяйте их.

Прежде чем начать форматировать свою тезис, сначала напишите и сохраните содержимое в виде отдельного текстового файла. Не используйте жесткие табуляции и ограничьте использование жестких возвратов только одним возвратом в конце абзаца. Не добавляйте никакой нумерации страниц в любом месте тезиса. 

\textbf{Авторы и аффилированность:}
Шаблон разработан таким образом, чтобы принадлежность автора не повторялась каждый раз для нескольких авторов одной и той же принадлежности. Пожалуйста, сохраняйте вашу принадлежность как можно более краткой (например, не делайте различий между отделами одной и той же организации).

\textbf{Правила оформления математических формул:}
Нумеруйте уравнения последовательно, указывая номера уравнений в скобках вровень с правым полем, как в~(\ref{Eq_1}). Чтобы сделать уравнения более компактными, можно использовать косую черту ( / ), функцию exp или соответствующие показатели степени. Выделяйте курсивом римские символы для величин и переменных, но не греческие символы. Используйте короткое тире (--) вместо дефиса для знака минус. Используйте скобки, чтобы избежать двусмысленности в знаменателях. 
\begin{equation}
\label{Eq_1}
\lambda_i = \lim \frac{1}{p} \sum_{t=1}^p \ln \frac{|w_i (t)|}{|w_i (t-1)|}
\end{equation}

Пожалуйста, установите в Microsoft Equation следующие шрифты: Regular~--- 12~pt, Large index~--- 7~pt, Small index~--- 5~pt, Large symbol~--- 18~pt, Small Symbol~--- 12~pt.

Обратите внимание, что уравнение центрируется с помощью центральной табуляции. Убедитесь, что символы в вашем уравнении определены до или сразу после уравнения. Используйте ``~(\ref{Eq_1})'', а не ``Eq.~(\ref{Eq_1})''или ``equation~(\ref{Eq_1})'', за исключением начала предложения:
``Equation~(\ref{Eq_1}) is \ldots{}''.

\textbf{Правила оформления ссылок:}
Нумеруйте цитаты последовательно в квадратных скобках \cite{1}. Пунктуация предложения следует за скобками\cite{1}. Просто ссылайтесь на номер ссылки, как в\cite{2}. Не используйте ``Ref.\cite{3}'' или ``reference\cite{3}'', за исключением начала предложения: ``Reference\cite{4} was the first \ldots{}''

Грамматически их можно рассматривать как номера сносок, например, как показано Адян Сергей\cite{2}; как упоминалось ранее\cite{1, 2, 3, 4, 5}; Линдон и Шепп\cite{5}; Кузнецов и др\cite{4}.

\textbf{Благодарности:}
Заголовки «Ссылки» и «Благодарности» не нумеруются. Это просто основные заголовки без меток, независимо от того, пронумерованы ли другие заголовки в статьях.
Размещение «Благодарности» происходит после окончательного текста статьи, непосредственно перед разделом «Ссылки» и после любого(их) Приложения(й).

\begin{thebibliography}{00}\label{ref:ref}
\bibitem{1}
Горюшкин А. П., Амальгамированные свободные произведения групп, ДВФУ, Владивосток, 2012, 158 с.
\bibitem{2}
Адян С. И., “Алгоритмическая неразрешимость проблем распознавания некоторых свойств групп”, Докл. АН СССР, 103:4 (1955), 533–535.
\bibitem{3}
Goryushkin A. P., “Imbedding of countable groups in 2-generated simple groups”, Mathematical Notes, 16:2 (1974), 725–727.
\bibitem{4}
Кузнецов А. В., “Алгоритмы как операции в алгебраических системах”, УМН, 13:3 (1958), 240–241.
\bibitem{5}
Линдон Р., Шупп П., Комбинаторная теория групп, Мир, М., 1980, 448 с.
\end{thebibliography}
\end{document}



